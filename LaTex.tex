%%%%%%%%%%%%%%%%%%%%%%%%%%%%%%%%%%%%%%%%%%%%%%%%%%%%%%%%%%%%
%%%%%%%%%%%%%%%%%%%%%%%%%%%%%%%%%%%%%%%%%%%%%%%%%%%%%%%%%%%%
%%%%%%%%%%%%%%%%%%%%%%%%%%%%%%%%%%%%%%%%%%%%%%%%%%%%%%%%%%%%
%%%%%%%%%%%%%%%%%%%%%%%%%%%%%%%%%%%%%%%%%%%%%%%%%%%%%%%%%%%%
%%%%%%%%%%%%%%%%%%%%%%%%%%%%%%%%%%%%%%%%%%%%%%%%%%%%%%%%%%%%
\documentclass[12pt]{article}
\usepackage{epsfig}
\usepackage{times}
\renewcommand{\topfraction}{1.0}
\renewcommand{\bottomfraction}{1.0}
\renewcommand{\textfraction}{0.0}
\setlength {\textwidth}{6.6in}
\hoffset=-1.0in
\oddsidemargin=1.00in
\marginparsep=0.0in
\marginparwidth=0.0in                                                            \usepackage{amsmath} 
\usepackage{fixltx2e}
\setlength {\textheight}{9.0in}
\voffset=-1.00in
\topmargin=1.0in
\headheight=0.0in
\headsep=0.00in
\footskip=0.50in                                         
%\setcounter{page}{1}
\begin{document}
\def\pos{\medskip\quad}
\def\subpos{\smallskip \qquad}
\newfont{\nice}{cmr12 scaled 1250}
\newfont{\name}{cmr12 scaled 1080}
\newfont{\swell}{cmbx12 scaled 800}

%%%%%%%%%%%%%%%%%%%%%%%%%%%%%%%%%%%%%%%%%%%%%%%%%%%%%%%%%%%%
\begin{center}
{\large\bf
PHYSICS  20323/60323: Fall 2019 - LaTeX Example}\\
%%%%%%%%%%%%%%%%%%%%%%%%%%%%%%%%%%%%%%%%%%%%%%%%%%%%%%%%%%%%
%%%%%%%%%%%%%%%%%%%%%%%%%%%%%%%%%%%%%%%%%%%%%%%%%%%%%%%%%%%%
\end{center}
\begin{enumerate}
\item {Consider a particle confined in a two-dimensional infinite square well}
\[V(x,y)= \bigg\{ \frac{0}{\infty}\begin{array}{cc}
    , & 0\leq x \leq a   ,   0 < y < a \\ 
   , & otherwise \\
   \end{array}
\]
The eigenfunctions have the form:
\begin{align*}
    \large{\Psi(x,y)= \frac{2}{a}\sin{\left( \frac{n\pi x}{a}\right)}\sin{\left(\frac{m\pi y}{a}\right)}}
\end{align*}
with the corresponding energies being given by:
\begin{center}
     \large{$E_{nm}=(n^{2}+m^{2})$} \Large{$\frac{\pi^{2}\hbar^{2}}{2ma^2}$}
\end{center}
\begin{enumerate}
    \item (5 points) What are the levels of degeneracy of the five lowest energy values?
    \item (5 points) Consider a perturbation given by:
\end{enumerate}
\begin{center}
    \large{$\hat{H}^{\prime}=a^{2}V_{0}\delta\left(x-\frac{a}{2}\right)\delta\left(y-\frac{a}{2}\right)$}
\end{center}
\hspace{0.75cm} Calculate the first order correction to the ground state energy. 
\item {\bf The following questions refer to stars in the Table below.}\\
Note: There may be multiple answers.\\
\\
\begin{tabular}{ |c|c|c|c|c|c| } 
 \hline
 Name & Mass & Luminosity & Lifetime & Temperature & Radius \\ \hline
 Zeta & 60. \textit{M\textsubscript{sun}} & 10\textsuperscript{6} \textit{L\textsubscript{sun}} & 8.0 $\times$10\textsuperscript{5} years &  &  \\ \hline
 Epsilon & 6.0 \textit{M\textsubscript{sun}} & 10\textsuperscript{3} \textit{L\textsubscript{sun}} &  & 20,000 K &  \\ \hline
 Delta & 2.0 \textit{M\textsubscript{sun}} &  & 5.0 $\times$10\textsuperscript{8} years &  & 2 \textit{R\textsubscript{sun}}\\ \hline
 Beta & 1.3 \textit{M\textsubscript{sun}} & 3.5  \textit{L\textsubscript{sun}} &  &  &  \\ \hline
 Alpha & 1.0 \textit{M\textsubscript{sun}} &  &  &  & 1 \textit{R\textsubscript{sun}} \\ \hline
 Gamma & 0.7 \textit{M\textsubscript{sun}} &  & 4.5 $\times$10\textsuperscript{10} years & 5000 K &  \\ \hline
\end{tabular}
\vskip 0.2in
\begin{enumerate}
    \item (4 points) Which of these stars will produce a planetary nebula at the end of their life
    \vskip 0.5in
    \item (4 points) Elements heavier than Carbon will be produced in which stars?
\end{enumerate}
\end{enumerate}
%%%%%%%%%%%%%%%%%%%%%%%%%%%%%%%%%%%%%%%%%%%%%%%%%%%%%%%%%%%%
\end{document}
